\documentclass[pdftex,11pt,letter]{article}
\usepackage{appendix} %http://www.tex.ac.uk/cgi-bin/texfaq2html?label=appendix
\usepackage{amsmath,amssymb,latexsym,float,epsfig}
\usepackage{framed,color,url,fancybox,fullpage,booktabs,subfigure,wrapfig,chngpage,setspace}
\newcommand{\ssection}[1]{\section[#1]{\centering\normalfont\scshape #1}}
\newcommand{\ssubsection}[1]{\subsection[#1]{\raggedright\normalfont\itshape #1}}
\newcommand\norm[1]{\left\lVert#1 \right\rVert}
\newcommand{\e}{\mathrm{e}}
\newcommand{\pderiv}[2]{\frac{\partial #1}{\partial #2}}
%\numberwithin{equation}{section}
%\numberwithin{figure}{section}
%\graphicspath{{./figures/}}
\usepackage{epstopdf,fancyvrb,cite,hyperref,jvlisting}
%\usepackage[options]{mcode}

\usepackage{pdflscape}%http://texblog.org/2007/11/10/landscape-in-latex/

% Define commands 
\newcommand{\half}{\ensuremath{\frac{1}{2}}}
\newcommand{\bea}{\begin{eqnarray}}
\newcommand{\eea}{\end{eqnarray}}
\newcommand{\beq}{\begin{equation}}
\newcommand{\eeq}{\end{equation}}
\newcommand{\bdm}{\begin{displaymath}}
\newcommand{\edm}{\end{displaymath}}
\newcommand{\etal}[0]{{\em et al.}}
\newcommand{\pd}[2]{\dfrac{\partial #1}{\partial #2}}
\newcommand{\pf}[2]{\dfrac{d #1}{d #2}}
\newcommand{\pdt}[2]{\dfrac{\partial^2 #1}{\partial #2^2}}
\newcommand{\pft}[2]{\dfrac{d^2 #1}{d #2^2}}
\newcommand{\pdtno}[2]{\dfrac{\partial^2 #1}{\partial #2}}
\newcommand{\pdd}[3]{\dfrac{\partial^2 #1}{\partial #2 \partial #3}}
\newcommand{\pff}[3]{\dfrac{d^2 #1}{d #2 d #3}}


\renewcommand\floatpagefraction{0.99}
\renewcommand\topfraction{0.99}
\renewcommand\bottomfraction{0.99}
\renewcommand\textfraction{0.0}

\usepackage{titling}
\setlength{\droptitle}{-1in}   % This is your set screw

% For \url{SOME_URL}, links SOME_URL to the url SOME_URL
\providecommand*\url[1]{\href{#1}{#1}}
% Same as above, but pretty-prints SOME_URL in teletype fixed-width font
\renewcommand*\url[1]{\href{#1}{\texttt{#1}}}

% For \email{ADDRESS}, links ADDRESS to the url mailto:ADDRESS
\providecommand*\email[1]{\href{mailto:#1}{#1}}

\usepackage{cancel}
\usepackage[margin=0.75in]{geometry}

\usepackage{multicol}
\setlength{\columnsep}{1cm}
 
\title{\textbf{A Diagonally Implicit Runge Kutta Method for First and Second Order Descriptor Systems}}
\author{{Komahan Boopathy} {and} {Graeme J. Kennedy}} \date{\today}

\begin{document}
\maketitle
\vspace{-0.25in}
\rule{\linewidth}{2pt}

\begin{multicols}{2}

%%%%%%%%%%%%%%%%%%%%%%%%%%%%%%%%%%%%%%%%%%%%%%%%%%%%%%%%%%%%%
%%%%%%%%%%%%%%%%%  NOMENCLATURE %%%%%%%%%%%%%%%%%%%%%%%%%%%%%
%%%%%%%%%%%%%%%%%%%%%%%%%%%%%%%%%%%%%%%%%%%%%%%%%%%%%%%%%%%%%

\section*{Nomenclature}
\begin{minipage}[b]{0.5\linewidth}\centering
\begin{tabular}{@{}lcl@{}}
$s$   && number of stages\\
$m$   && number of variables\\
$h$   && time step size \\
$i,j$ && stage indices \\
$k$   && time index \\
$c_j,b_j,a_{ij}$ && Butcher tableau entries \\
$q_k$ && state variables\\
$q_{k,i}$ && stage states \\
$R_i$  && $i-$th stage residual equation \\
$J_{ij}$  && $i,j-$th Jacobian entry \\
\end{tabular}
\end{minipage}

%%%%%%%%%%%%%%%%%%%%%%%%%%%%%%%%%%%%%%%%%%%%%%%%%%%%%%%%%%%%%
%%%%%%%%%%%%%%%%%  IMPLICIT RUNGE KUTTA %%%%%%%%%%%%%%%%%%%%%
%%%%%%%%%%%%%%%%%%%%%%%%%%%%%%%%%%%%%%%%%%%%%%%%%%%%%%%%%%%%%

\section{Descriptor Systems}

First and second order descriptor systems are considered in this
derivation.

\subsection{First Order System}

\paragraph{Governing Equations:}

Consider a system of first order differential equation in descriptor
form

\begin{equation}\label{first_order_eqn_desc}
  \mathbf{R}(t, \mathbf{q}, \mathbf{\dot{q}}) = 0
\end{equation}
Here $\mathbf{q}$ is a vector of $m$ \textit{state variables} and
consequently $\mathbf{R}$ is a system of $m$ governing equations.

\paragraph{Stage Equations:}

We write the stage equations in the following form

\begin{equation}\label{stage_eqn_first}
  \mathbf{q}_{k,i} = \mathbf{q}_{k} + h \sum_{j=1}^s a_{ij}
  \mathbf{\dot{q}}_{k,j} \quad i = 1,\ldots,s
\end{equation}
In implicit or descriptor form, this takes the representation
\begin{equation}\label{stage_eqn_desc}
  \mathbf{R}\left(t_{k,i}, \mathbf{q}_{k,i}, \mathbf{\dot{q}}_{k,i}\right) = 0
\end{equation}
%\paragraph{Stage values:}
%\begin{equation}
%  \begin{split}
%    \mathbf{q}_{k,i} & = \mathbf{q}_{k} + h \sum_{j=1}^s a_{ij}
%    \mathbf{\dot{q}}_{k,j} \quad i = 1,\ldots,s \\
%  \end{split}
%\end{equation}
If one uses DIRK, Eq~\ref{stage_eqn_desc} can be solved
successively $i.e.$ stage--after--stage. In one were to use a fully
implicit Runge-Kutta, a coupled system needs to be solved, to get the
stage values. In this work, we use prefer to use DIRK owing to its
superiority in ease of implementation and computational savings due to
the structure of the corresponding Butcher tableau. The solution of
the nonlinear stage equations is described below.

\paragraph{Newton's method:}

The $m \times m$ system of governing equations~\eqref{stage_eqn_first}
are solved using Newton's method iteratively for each stage $i$.

\begin{equation}
\left[\pd{\mathbf{R}}{\mathbf{\dot{q}}} + h a_{ii}
  \pd{\mathbf{R}}{\mathbf{q}}\right] \Delta\mathbf{\dot{q}} = -\mathbf{R}
\end{equation}

\paragraph{Newton update:}
The Newton updates are carried out as shown below
\begin{equation}\label{newton_update}
  \begin{split}
    \mathbf{\dot{q}}_{k,i}^{n+1} & = \mathbf{\dot{q}}_{k,i}^{n} +
    \Delta\mathbf{\dot{q}}_{k,i}^{n} \\
    \mathbf{q}_{k,i}^{n+1} & = \mathbf{q}_{k,i}^n + h a_{ii}
    \mathbf{\dot{q}}_{k,i}^n \\
  \end{split}
\end{equation}
where $n$ refers to the Newton iteration number.

\paragraph{Time Marching:}

Once the stage derivatives $\mathbf{\dot{q}}_{k,j}$ are found, they can
be used to find the next value of the state as follows
\begin{equation}
  \mathbf{q}_{k+1} = \mathbf{q}_{k} +h \sum_{j=1}^s b_j \mathbf{\dot{q}}_{k,j}
\end{equation}

\subsection{Second Order System}

\begin{equation}
  \mathbf{R}\left(t_{k,i}, \mathbf{q}_{k,i}, \mathbf{\dot{q}}_{k,i},
  \mathbf{\ddot{q}}_{k,i}\right) = 0
\end{equation}
Let $\mathbf{v}=\mathbf{\dot{q}}$, therefore $\mathbf{\dot{v}}=\mathbf{\ddot{q}}$
Therefore, the governing equations are
\begin{equation}
  \mathbf{R}\left(t_{k,i}, \mathbf{q}_{k,i}, \mathbf{v}_{k,i}, \mathbf{\dot{v}}_{k,i}\right) = 0
\end{equation}
\begin{equation}
  \mathbf{q}_{k,i} = \mathbf{q}_{k} + h \sum_{j=1}^s a_{ij} \mathbf{\dot{q}}_{k,j} \quad i = 1,\ldots,s 
\end{equation}

\begin{equation}
  \mathbf{v}_{k,i} = \mathbf{v}_{k} + h \sum_{j=1}^s a_{ij} \mathbf{\dot{v}}_{k,j} \quad i = 1,\ldots,s 
\end{equation}
Rearranging, to get the stage equations
\begin{equation}
  \mathbf{q}_{k,i} - \mathbf{q}_{k} - h \sum_{j=1}^s a_{ij}
  \mathbf{\dot{q}}_{k,j} = 0 \quad i = 1,\ldots,s
\end{equation}
\begin{equation}
  \mathbf{{v}}_{k,i} - \mathbf{v}_{k} - h \sum_{j=1}^s a_{ij}
  \mathbf{\dot{v}}_{k,j} = 0 \quad i = 1,\ldots,s
\end{equation}

\begin{equation}
  \begin{bmatrix}
    \mathbf{\pd{R}{\dot{q}}} & \mathbf{\pd{R}{\dot{v}}} \\
    \mathbf{I} & -h a_{ii}\mathbf{I}\\
  \end{bmatrix}   \begin{Bmatrix}
    \Delta{\mathbf{\dot{q}}} \\
    \Delta{\mathbf{\dot{v}}} \\
  \end{Bmatrix} =    \begin{Bmatrix}
    -\mathbf{R} \\ \mathbf{0}
  \end{Bmatrix}
\end{equation}
This is a $2n \times 2n$ system. We can write the system as follows

\begin{equation}
 \mathbf{\pd{R}{\dot{q}}} \Delta \mathbf{\dot{q}} +
 \mathbf{\pd{R}{\dot{v}}} \Delta \mathbf{\dot{v}} = -\mathbf{R}
\end{equation}
\begin{equation}
  \Delta \mathbf{\dot{q}} = h a_{ii} \Delta \mathbf{\dot{v}}
\end{equation}
By substitution for $\Delta \mathbf{\dot{q}} $ we can write
\begin{equation}
  \mathbf{J} = \pd{\mathbf{R}}{\mathbf{\dot{v}}_{i}} + h a_{ii}
  \pd{\mathbf{R}}{\mathbf{\dot{q}}_{i}}
\end{equation}
Using the relations
\begin{equation}
  \pd{\mathbf{R}}{\mathbf{\dot{q}}_{i}} = h a_{ii} \pd{\mathbf{R}}{\mathbf{{q}}}
\end{equation}
\begin{equation}
  \pd{\mathbf{R}}{\mathbf{\dot{v}}_{i}} =  \pd{\mathbf{R}}{\mathbf{\ddot{q}}_{i}} + h a_{ii} \pd{\mathbf{R}}{\mathbf{\dot{q}}} 
\end{equation}

\begin{equation}
  \mathbf{J} = \pd{\mathbf{R}}{\mathbf{\ddot{q}}} + h a_{ii}
  \pd{\mathbf{R}}{\mathbf{\dot{q}}} + h^2 a_{ii}^2
  \pd{\mathbf{R}}{\mathbf{{q}}}
\end{equation}
We solve the following system with Newton's method.
\begin{equation}
 \mathbf{J}
 \left(\mathbf{q}_{k,i},\mathbf{\dot{q}}_{k,i},\mathbf{\ddot{q}}_{k,i}\right)
 \Delta\mathbf{\ddot{q}} = -\mathbf{R}
 \left(\mathbf{q}_{k,i},\mathbf{\dot{q}}_{k,i},\mathbf{\ddot{q}}_{k,i}\right)
\end{equation}

\paragraph{Stage values:}
The stage values for the above system are given by the following relations
\begin{equation}
  \begin{split}
    \mathbf{q}_{k,i}  & = \mathbf{q}_{k} + h \sum_{j=1}^s a_{ij} \mathbf{\dot{q}}_{k,j} \quad i = 1,\ldots,s \\
    \mathbf{\dot{q}}_{k,i}  & = \mathbf{\dot{q}}_{k} + h \sum_{j=1}^s a_{ij} \mathbf{\ddot{q}}_{k,j} \quad i = 1,\ldots,s \\
  \end{split}
\end{equation}

\paragraph{Newton update:}
The following are the update formulae for the solution of the non-linear problem
\begin{equation}\label{newton_update2}
  \begin{split}
    \mathbf{\ddot{q}}_{k,i}^{n+1} & = \mathbf{\ddot{q}}_{k,i}^{n} +
    \Delta\mathbf{\ddot{q}}_{k,i}^{n} \\
    \mathbf{\dot{q}}_{k,i}^{n+1} & = \mathbf{\dot{q}}_{k,i}^n + h a_{ii}
    \mathbf{\ddot{q}}_{k,i}^n \\
    \mathbf{{q}}_{k,i}^{n+1} & = \mathbf{{q}}_{k,i}^n + h^2 a_{ii}^2
    \mathbf{\ddot{q}}_{k,i}^n \\
  \end{split}
\end{equation}
where $n$ refers to the Newton iteration number.

\paragraph{Time marching:}
We march in time using the following relation
\begin{equation}
  \begin{split}
  \mathbf{q}_{k+1} = \mathbf{q}_{k} +h \sum_{j=1}^s b_j \mathbf{\dot{q}}_{k,j} \\
  \mathbf{\dot{q}}_{k+1} = \mathbf{\dot{q}}_{k} +h \sum_{j=1}^s b_j \mathbf{\ddot{q}}_{k,j} \\
  \end{split}
\end{equation}

%%%%%%%%%%%%%%%%%%%%%%%%%%%%%%%%%%%%%%%%%%%%%%%%%%%%%%%%%%%%%%%%%%%%%%%
%  RESULTS
%%%%%%%%%%%%%%%%%%%%%%%%%%%%%%%%%%%%%%%%%%%%%%%%%%%%%%%%%%%%%%%%%%%%%%%

\section{Results}

\subsection{First Order Systems}

In this section, we compare our descriptor scheme with that of
conventional DIRK method of solving the first order differential
equations in explicit form.

\subsubsection{Test Functions}
 

\paragraph{Test Case 1}

In explicit form
\begin{equation}
  \dot{q} = f(q) = \sin(q)
\end{equation}
In descriptor form
\begin{equation}
  R(q, \dot{q}) = \dot{q}- \sin(q) = 0
\end{equation}

\paragraph{Test Case 2}

In explicit form
\begin{equation}
  \begin{split}
%  \dot{q_1} = f_1(q) = q_2
  \end{split}
\end{equation}
In descriptor form
\begin{equation}
  R(q, \dot{q}) = \dot{q}- \sin(q) = 0
\end{equation}

\paragraph{Comparison of Solutions}

\paragraph{Comparison of Order of Accuracy}

\end{multicols}
\end{document}

\paragraph{Residual of Stage Equations:}

The residual of the stage equations:
\begin{equation}
  F_i = q_{k,i} - q_{k} - h \sum_{j=1}^s a_{ij} \dot{q}_{k,j} \quad i = 1,\ldots,s 
\end{equation}

\paragraph{Jacobian of Stage Equations:}

The Jacobian of the stage equations:
\begin{equation}
  \begin{split}
    J_{ii} = 1 - h a_{ii} \pd{\dot{q}(t_{k,i}, q_{k,i})}{q} \quad \forall i = j \\
    J_{ij} = - h a_{ij} \pd{\dot{q}(t_{k,j}, q_{k,j})}{q} \quad \forall i \ne j
  \end{split}
\end{equation}



From an implementation point of view, the only changes occur within
user implemented  routines, compared to the explicit RK scheme:
\begin{enumerate}
\item \texttt{compute\_stage\_residual} and
\item \texttt{compute\_stage\_jacobian}
\end{enumerate}

\section{Results}

\begin{figure}[H]
  \begin{minipage}{0.33\linewidth}
    \includegraphics[width=\linewidth]{{dae_descriptor_explicit_stage1_h0.1}.eps}
  \end{minipage}
  \begin{minipage}{0.33\linewidth}
    \includegraphics[width=\linewidth]{{dae_descriptor_explicit_stage2_h0.1}.eps}
  \end{minipage}
  \begin{minipage}{0.33\linewidth}
    \includegraphics[width=\linewidth]{{dae_descriptor_explicit_stage3_h0.1}.eps}
  \end{minipage}
  \begin{minipage}{0.33\linewidth}
    \includegraphics[width=\linewidth]{{dae_descriptor_explicit_stage1_h0.25}.eps}
  \end{minipage}
  \begin{minipage}{0.33\linewidth}
    \includegraphics[width=\linewidth]{{dae_descriptor_explicit_stage2_h0.25}.eps}
  \end{minipage}
  \begin{minipage}{0.33\linewidth}
    \includegraphics[width=\linewidth]{{dae_descriptor_explicit_stage3_h0.25}.eps}
  \end{minipage}
  \begin{minipage}{0.33\linewidth}
    \includegraphics[width=\linewidth]{{dae_descriptor_explicit_stage1_h0.5}.eps}
  \end{minipage}
  \begin{minipage}{0.33\linewidth}
    \includegraphics[width=\linewidth]{{dae_descriptor_explicit_stage2_h0.5}.eps}
  \end{minipage}
  \begin{minipage}{0.33\linewidth}
    \includegraphics[width=\linewidth]{{dae_descriptor_explicit_stage3_h0.5}.eps}
  \end{minipage}
  \caption{A comparison of solutions obtained by solving descriptor
    system and explicit system of first order differential equations
    for different step sizes and number of intermediate RK stages for the function $\dot{q}=\sin{q}+\cos{t}$.}
\end{figure}

\begin{table}[h]
\caption{Number of function and jacobian calls made \textbf{during each time step}}
\medskip
\centering 
\begin{tabular}{c | c c}
\hline
Method & DIRK1-Explicit & DIRK1-Descriptor  \\
\hline
F  & 4  & 11\\
FG & 4  & 11\\ 
\hline
Method & DIRK2-Explicit & DIRK2-Descriptor  \\
\hline
F  & 14 & 43 \\
FG & 18 & 59 \\ 
\hline
Method & DIRK3-Explicit & DIRK3-Descriptor  \\
\hline
F  & 32 & 105 \\
FG & 56 & 184 \\ 
\hline
\end{tabular}
\label{tab:com_cost}
\end{table}
This difference in the number of calls is due to the fact that the
descriptor system takes \underline{three times} as many Newton
iterations as the explicit formulation. The explicit formulation takes
about 5 Newton iterations whereas the solution of the descriptor
nonlinear stage equations take 15 iterations.



%%%%%%%%%%%%%%%%%%%%%%%%%%%%%%%%%%%%%%%%%%%%%%%%%%%%%%%%%%%%%
%%%%%%%%%%%%%%%%%  EXPLICIT RUNGE KUTTA %%%%%%%%%%%%%%%%%%%%%
%%%%%%%%%%%%%%%%%%%%%%%%%%%%%%%%%%%%%%%%%%%%%%%%%%%%%%%%%%%%%

\section{Governing Equations in Explicit Form}

Consider a first order differential equation  of the form
\begin{equation}
  \pf{q}{t} = f(t, q)
\end{equation}
In IRK, one needs to solve a system of nonlinear stage equations at
each time step. 
\paragraph{Stage Equations:}
The stage equations to be solved are as follows:
\begin{equation}\label{stage_explicit}
  q_{k,i} = q_{k} + h \sum_{j=1}^s a_{ij} f(t_{k,j}, q_{k,j}) \quad i = 1,\ldots,s 
\end{equation}
Here, $t_{k,j}$ is provided by the formula:
\begin{equation}
  t_{k,j} = t_k + c_j h
\end{equation}
\paragraph{Residual of Stage Equations:}
 The residual of the stage equations:
\begin{equation}
  F_i = q_{k,i} - q_{k} - h \sum_{j=1}^s a_{ij} f(t_{k,j}, q_{k,j}) \quad i = 1,\ldots,s 
\end{equation}
\paragraph{Jacobian of Stage Equations:}
The jacobian of the stage equations:
\begin{equation}
  \begin{split}
    J_{ii} = 1 - h a_{ii} \pd{f(t_{k,i}, q_{k,i})}{q} \quad \forall i = j \\
    J_{ij} = - h a_{ij} \pd{f(t_{k,j}, q_{k,j})}{q} \quad \forall i \ne j
  \end{split}
\end{equation}
Note that we may save a few function and derivative evaluations when
the corresponding Butcher tableau entries $a_{ij}$ are zero. Along the
same line of discussion, DIRK offers greater computational savings.

\paragraph{Time Marching:}
Once the stage states $q_{k,i}$ are found, they can be used to find the next
value of the state as follows:
\begin{equation}
  q_{k+1} = q_{k} + h \sum_{i=1}^s b_i f(t_{k,i},q_{k,i})
\end{equation}

\subsection{Two-Stage IRK Example}

Consider an example of two-stage IRK. The stage residual is:
\begin{equation}\label{eq:sr2}
  \begin{split}
    \mathbf{F} & = \left\{
    \begin{array}{rr} q_{k,1} - q_{k} - h \sum_{j=1}^s a_{1j} f(t_{k,j}, q_{k,j}) \\
      q_{k,2} - q_{k} - h \sum_{j=1}^s a_{2j} f(t_{k,j}, q_{k,j})
    \end{array} \right\}\\
    & = \left\{
    \begin{array}{rr} q_{k,1} - q_{k} -  a_{11} h f(t_{k,1}, q_{k,1}) + a_{12} h f(t_{k,2}, q_{k,2})\\
      q_{k,2} - q_{k} - a_{21} h f(t_{k,1}, q_{k,1}) + a_{22} h f(t_{k,2}, q_{k,2})
    \end{array} \right\}
  \end{split}
\end{equation}
We need to solve for $q_{k,1}$ and $q_{k,2}$, differentiating $\mathbf{F}$ we get
the stage jacobian matrix:
\begin{equation}\label{eq:sj2}
  \mathbf{J} =  \begin{bmatrix}
    1 - h a_{11} \pd{f(t_{k,1}, q_{k,1})}{q} & - h a_{12} \pd{f(t_{k,2}, q_{k,2})}{q}\\
    - h a_{21} \pd{f(t_{k,1}, q_{k,1})}{q}   & 1 - h a_{22} \pd{f(t_{k,2}, q_{k,2})}{q}
  \end{bmatrix}
\end{equation}
Eqs.~\eqref{eq:sr2} and ~\eqref{eq:sj2} are to be solved for the state
variable values $q_{k,1}$ and $q_{k,2}$, using iterative methods such as the
Newton's method.

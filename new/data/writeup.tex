\documentclass[pdftex,11pt,letter]{article}
\usepackage{appendix} %http://www.tex.ac.uk/cgi-bin/texfaq2html?label=appendix
\usepackage{amsmath,amssymb,latexsym,float,epsfig}
\usepackage{framed,color,url,fancybox,fullpage,booktabs,subfigure,wrapfig,chngpage,setspace}
\newcommand{\ssection}[1]{\section[#1]{\centering\normalfont\scshape #1}}
\newcommand{\ssubsection}[1]{\subsection[#1]{\raggedright\normalfont\itshape #1}}
\newcommand\norm[1]{\left\lVert#1 \right\rVert}
\newcommand{\e}{\mathrm{e}}
\newcommand{\pderiv}[2]{\frac{\partial #1}{\partial #2}}
%\numberwithin{equation}{section}
%\numberwithin{figure}{section}
%\graphicspath{{./figures/}}
\usepackage{epstopdf,fancyvrb,cite,hyperref,jvlisting}
%\usepackage[options]{mcode}

\usepackage{pdflscape}%http://texblog.org/2007/11/10/landscape-in-latex/

% Define commands 
\newcommand{\half}{\ensuremath{\frac{1}{2}}}
\newcommand{\bea}{\begin{eqnarray}}
\newcommand{\eea}{\end{eqnarray}}
\newcommand{\beq}{\begin{equation}}
\newcommand{\eeq}{\end{equation}}
\newcommand{\bdm}{\begin{displaymath}}
\newcommand{\edm}{\end{displaymath}}
\newcommand{\etal}[0]{{\em et al.}}
\newcommand{\pd}[2]{\dfrac{\partial #1}{\partial #2}}
\newcommand{\pf}[2]{\dfrac{d #1}{d #2}}
\newcommand{\pdt}[2]{\dfrac{\partial^2 #1}{\partial #2^2}}
\newcommand{\pft}[2]{\dfrac{d^2 #1}{d #2^2}}
\newcommand{\pdtno}[2]{\dfrac{\partial^2 #1}{\partial #2}}
\newcommand{\pdd}[3]{\dfrac{\partial^2 #1}{\partial #2 \partial #3}}
\newcommand{\pff}[3]{\dfrac{d^2 #1}{d #2 d #3}}


\renewcommand\floatpagefraction{0.99}
\renewcommand\topfraction{0.99}
\renewcommand\bottomfraction{0.99}
\renewcommand\textfraction{0.0}

\usepackage{titling}
\setlength{\droptitle}{-1in}   % This is your set screw

% For \url{SOME_URL}, links SOME_URL to the url SOME_URL
\providecommand*\url[1]{\href{#1}{#1}}
% Same as above, but pretty-prints SOME_URL in teletype fixed-width font
\renewcommand*\url[1]{\href{#1}{\texttt{#1}}}

% For \email{ADDRESS}, links ADDRESS to the url mailto:ADDRESS
\providecommand*\email[1]{\href{mailto:#1}{#1}}

\usepackage{cancel}
\usepackage[margin=0.75in]{geometry}
\title{\textbf{Implicit Runge Kutta Method}}
\author{Komahan Boopathy~~\url{komahan@gatech.edu}} \date{\today}

\begin{document}
\maketitle
\vspace{-0.25in}
\rule{\linewidth}{2pt}


%%%%%%%%%%%%%%%%%%%%%%%%%%%%%%%%%%%%%%%%%%%%%%%%%%%%%%%%%%%%%
%%%%%%%%%%%%%%%%%  NOMENCLATURE %%%%%%%%%%%%%%%%%%%%%%%%%%%%%
%%%%%%%%%%%%%%%%%%%%%%%%%%%%%%%%%%%%%%%%%%%%%%%%%%%%%%%%%%%%%

\section*{Nomenclature}
\begin{minipage}[b]{0.5\linewidth}\centering
\begin{tabular}{@{}lcl@{}}
$s$   && number of stages\\
$h$   && time step size \\
$i,j$ && stage indices \\
$k$   && time index \\
$c_j,b_j,a_{ij}$ && Butcher tableau entries \\
$q_k$ && state variable value at previous/initial time \\
$q_{k,i}$ && stage state variable values relative to previous/initial time \\
$R_i$  && $i-$th stage residual equation \\
$J_{ij}$  && $i,j-$th stage jacobian entry \\
\end{tabular}
\end{minipage}

%%%%%%%%%%%%%%%%%%%%%%%%%%%%%%%%%%%%%%%%%%%%%%%%%%%%%%%%%%%%%
%%%%%%%%%%%%%%%%%  IMPLICIT RUNGE KUTTA %%%%%%%%%%%%%%%%%%%%%
%%%%%%%%%%%%%%%%%%%%%%%%%%%%%%%%%%%%%%%%%%%%%%%%%%%%%%%%%%%%%

\section{Governing Equations in Implicit Form}

Consider first order differential equation in descriptor form
\begin{equation}
  \mathbf{R}(t_k, q_k, \dot{q}_k) = 0
\end{equation}

\paragraph{Stage Equations:}
The key is to write Eq.~\ref{stage_explicit} in the following form
\begin{equation}\label{stage_implicit}
  q_{k,i} = q_{k} + h \sum_{j=1}^s a_{ij} \dot{q}_{k,j} \quad i = 1,\ldots,s 
\end{equation}
The governing equation in descriptor form for different stages $i$ at the $k-$th time-step takes the form:
\begin{equation}
  \mathbf{R}\left(t_{k,i}, q_{k,i}, \dot{q}_{k,i}\right) = 0
\end{equation}
Using Eq.~\eqref{stage_implicit}, this becomes
\begin{equation}
  \mathbf{R}\left(t_{k,i}, q_k + h \sum_{j=1}^s a_{ij}\dot{q}_{k,j}, \dot{q}_{k,i}\right) = 0
\end{equation}
In this equation we need to solve for $q_{k,i}$ values. We use
Newtons's method to solve for these stage derivatives. To aid the
solution, we need the Jacobian of the governing equation.
\begin{equation}
  \begin{split}
    J = \pf{R}{\dot{q}} & = \pd{R}{\dot{q}} + \pd{R}{{q}}  \pd{{q}}{\dot{q}} \\
  & =  \pd{R}{\dot{q}} + h a_{ii} \pd{R}{q}\\
  \end{split}
\end{equation}

\beq
\mathbf{J} = \pd{\mathbf{R}\left(t_{k,i},{q}_{k,i},\dot{q}_{k,i}\right)}{\dot{q}} + h a_{ii} \pd{\mathbf{R}\left(t_{k,i},{q}_{k,i},\dot{q}_{k,i}\right)}{{q}}
\eeq
  

\paragraph{Time Marching:}

Once the stage derivatives $\dot{q}_{k,i}$ are found, they can be used
to find the next value of the state as follows:
\begin{equation}
  q_{k+1} = q_{k} + h \sum_{i=1}^s b_i \dot{q}_{k,i}
\end{equation}

\section{Second order system}

The governing equation in descriptor form for the $k-$th time-step
takes the form:
\begin{equation}
  \mathbf{R}\left(t_{k}, q_{k}, \dot{q}_{k},\ddot{q}_{k}\right) = 0
\end{equation}
%
%\paragraph{Stage Equations:}
%We write Eq.~\ref{stage_explicit} in the following form for this
%system
%\begin{equation}\label{stage_implicit}
%  q_{k,i} = q_{k} + h \sum_{j=1}^s a_{ij} \dot{q}_{k,j} + \frac{h^2}{2} \sum_{j=1}^s a_{ij} \ddot{q}_{k,j} \quad i = 1,\ldots,s 
%\end{equation}
%
%\begin{equation}\label{stage_implicit}
%  q_{k,i} = q_{k} + h \dot{q}_k + \frac{h^2}{2} \sum_{j=1}^s a_{ij} \ddot{q}_{k,j} \quad i = 1,\ldots,s 
%\end{equation}


\paragraph{Representation 1}

\begin{equation}\label{stage_implicit1}
  q_{k,i} = q_{k} + h \sum_{j=1}^s a_{ij} \dot{q}_{k,j} + \frac{h^2}{2} \sum_{j=1}^s a_{ij} \ddot{q}_{k,j} \quad i = 1,\ldots,s 
\end{equation}

\begin{equation}\label{stage_implicit2}
  \dot{q}_{k,i} = \dot{q}_{k} + h \sum_{j=1}^s a_{ij} \ddot{q}_{k,j} \quad i = 1,\ldots,s 
\end{equation}


\beq
\mathbf{R}\left(t_{k,i}, q_{k,i}, \dot{q}_{k,i},\ddot{q}_{k,i},     \right) = 0
\eeq



\paragraph{Representation 2}

\begin{equation}\label{stage_implicit1}
  q_{k,i} = q_{k} + h \sum_{j=1}^s a_{ij} \dot{q}_{k,j} \quad i = 1,\ldots,s 
\end{equation}

\begin{equation}\label{stage_implicit2}
  \dot{q}_{k,i} = \dot{q}_{k} + h \sum_{j=1}^s a_{ij} \ddot{q}_{k,j} \quad i = 1,\ldots,s 
\end{equation}


\beq
\mathbf{R}\left(t_{k,i}, q_{k,i}, \dot{q}_{k,i},\ddot{q}_{k,i},     \right) = 0
\eeq



\end{document}

\paragraph{Residual of Stage Equations:}

The residual of the stage equations:
\begin{equation}
  F_i = q_{k,i} - q_{k} - h \sum_{j=1}^s a_{ij} \dot{q}_{k,j} \quad i = 1,\ldots,s 
\end{equation}

\paragraph{Jacobian of Stage Equations:}

The jacobian of the stage equations:
\begin{equation}
  \begin{split}
    J_{ii} = 1 - h a_{ii} \pd{\dot{q}(t_{k,i}, q_{k,i})}{q} \quad \forall i = j \\
    J_{ij} = - h a_{ij} \pd{\dot{q}(t_{k,j}, q_{k,j})}{q} \quad \forall i \ne j
  \end{split}
\end{equation}



From an implementation point of view, the only changes occur within
user implemented  routines, compared to the explicit RK scheme:
\begin{enumerate}
\item \texttt{compute\_stage\_residual} and
\item \texttt{compute\_stage\_jacobian}
\end{enumerate}

\section{Results}

\begin{figure}[H]
  \begin{minipage}{0.33\linewidth}
    \includegraphics[width=\linewidth]{{dae_descriptor_explicit_stage1_h0.1}.eps}
  \end{minipage}
  \begin{minipage}{0.33\linewidth}
    \includegraphics[width=\linewidth]{{dae_descriptor_explicit_stage2_h0.1}.eps}
  \end{minipage}
  \begin{minipage}{0.33\linewidth}
    \includegraphics[width=\linewidth]{{dae_descriptor_explicit_stage3_h0.1}.eps}
  \end{minipage}
  \begin{minipage}{0.33\linewidth}
    \includegraphics[width=\linewidth]{{dae_descriptor_explicit_stage1_h0.25}.eps}
  \end{minipage}
  \begin{minipage}{0.33\linewidth}
    \includegraphics[width=\linewidth]{{dae_descriptor_explicit_stage2_h0.25}.eps}
  \end{minipage}
  \begin{minipage}{0.33\linewidth}
    \includegraphics[width=\linewidth]{{dae_descriptor_explicit_stage3_h0.25}.eps}
  \end{minipage}
  \begin{minipage}{0.33\linewidth}
    \includegraphics[width=\linewidth]{{dae_descriptor_explicit_stage1_h0.5}.eps}
  \end{minipage}
  \begin{minipage}{0.33\linewidth}
    \includegraphics[width=\linewidth]{{dae_descriptor_explicit_stage2_h0.5}.eps}
  \end{minipage}
  \begin{minipage}{0.33\linewidth}
    \includegraphics[width=\linewidth]{{dae_descriptor_explicit_stage3_h0.5}.eps}
  \end{minipage}
  \caption{A comparison of solutions obtained by solving descriptor
    system and explicit system of first order differential equations
    for different step sizes and number of intermediate RK stages for the function $\dot{q}=\sin{q}+\cos{t}$.}
\end{figure}

\begin{table}[h]
\caption{Number of function and jacobian calls made \textbf{during each time step}}
\medskip
\centering 
\begin{tabular}{c | c c}
\hline
Method & DIRK1-Explicit & DIRK1-Descriptor  \\
\hline
F  & 4  & 11\\
FG & 4  & 11\\ 
\hline
Method & DIRK2-Explicit & DIRK2-Descriptor  \\
\hline
F  & 14 & 43 \\
FG & 18 & 59 \\ 
\hline
Method & DIRK3-Explicit & DIRK3-Descriptor  \\
\hline
F  & 32 & 105 \\
FG & 56 & 184 \\ 
\hline
\end{tabular}
\label{tab:com_cost}
\end{table}
This difference in the number of calls is due to the fact that the
descriptor system takes \underline{three times} as many Newton
iterations as the explicit formulation. The explicit formulation takes
about 5 Newton iterations whereas the solution of the descriptor
nonlinear stage equations take 15 iterations.



%%%%%%%%%%%%%%%%%%%%%%%%%%%%%%%%%%%%%%%%%%%%%%%%%%%%%%%%%%%%%
%%%%%%%%%%%%%%%%%  EXPLICIT RUNGE KUTTA %%%%%%%%%%%%%%%%%%%%%
%%%%%%%%%%%%%%%%%%%%%%%%%%%%%%%%%%%%%%%%%%%%%%%%%%%%%%%%%%%%%

\section{Governing Equations in Explicit Form}

Consider a first order differential equation  of the form
\begin{equation}
  \pf{q}{t} = f(t, q)
\end{equation}
In IRK, one needs to solve a system of nonlinear stage equations at
each time step. 
\paragraph{Stage Equations:}
The stage equations to be solved are as follows:
\begin{equation}\label{stage_explicit}
  q_{k,i} = q_{k} + h \sum_{j=1}^s a_{ij} f(t_{k,j}, q_{k,j}) \quad i = 1,\ldots,s 
\end{equation}
Here, $t_{k,j}$ is provided by the formula:
\begin{equation}
  t_{k,j} = t_k + c_j h
\end{equation}
\paragraph{Residual of Stage Equations:}
 The residual of the stage equations:
\begin{equation}
  F_i = q_{k,i} - q_{k} - h \sum_{j=1}^s a_{ij} f(t_{k,j}, q_{k,j}) \quad i = 1,\ldots,s 
\end{equation}
\paragraph{Jacobian of Stage Equations:}
The jacobian of the stage equations:
\begin{equation}
  \begin{split}
    J_{ii} = 1 - h a_{ii} \pd{f(t_{k,i}, q_{k,i})}{q} \quad \forall i = j \\
    J_{ij} = - h a_{ij} \pd{f(t_{k,j}, q_{k,j})}{q} \quad \forall i \ne j
  \end{split}
\end{equation}
Note that we may save a few function and derivative evaluations when
the corresponding Butcher tableau entries $a_{ij}$ are zero. Along the
same line of discussion, DIRK offers greater computational savings.

\paragraph{Time Marching:}
Once the stage states $q_{k,i}$ are found, they can be used to find the next
value of the state as follows:
\begin{equation}
  q_{k+1} = q_{k} + h \sum_{i=1}^s b_i f(t_{k,i},q_{k,i})
\end{equation}

\subsection{Two-Stage IRK Example}

Consider an example of two-stage IRK. The stage residual is:
\begin{equation}\label{eq:sr2}
  \begin{split}
    \mathbf{F} & = \left\{
    \begin{array}{rr} q_{k,1} - q_{k} - h \sum_{j=1}^s a_{1j} f(t_{k,j}, q_{k,j}) \\
      q_{k,2} - q_{k} - h \sum_{j=1}^s a_{2j} f(t_{k,j}, q_{k,j})
    \end{array} \right\}\\
    & = \left\{
    \begin{array}{rr} q_{k,1} - q_{k} -  a_{11} h f(t_{k,1}, q_{k,1}) + a_{12} h f(t_{k,2}, q_{k,2})\\
      q_{k,2} - q_{k} - a_{21} h f(t_{k,1}, q_{k,1}) + a_{22} h f(t_{k,2}, q_{k,2})
    \end{array} \right\}
  \end{split}
\end{equation}
We need to solve for $q_{k,1}$ and $q_{k,2}$, differentiating $\mathbf{F}$ we get
the stage jacobian matrix:
\begin{equation}\label{eq:sj2}
  \mathbf{J} =  \begin{bmatrix}
    1 - h a_{11} \pd{f(t_{k,1}, q_{k,1})}{q} & - h a_{12} \pd{f(t_{k,2}, q_{k,2})}{q}\\
    - h a_{21} \pd{f(t_{k,1}, q_{k,1})}{q}   & 1 - h a_{22} \pd{f(t_{k,2}, q_{k,2})}{q}
  \end{bmatrix}
\end{equation}
Eqs.~\eqref{eq:sr2} and ~\eqref{eq:sj2} are to be solved for the state
variable values $q_{k,1}$ and $q_{k,2}$, using iterative methods such as the
Newton's method.
